\documentclass[12pt]{article}
\usepackage{enumerate}
\usepackage{fullpage}
\usepackage[brazilian]{babel}
\usepackage[num]{abntex2cite}
\usepackage[margin=1in,footskip=0.25in]{geometry}
\usepackage[utf8]{inputenc}
\usepackage{graphicx}
\usepackage{hyperref}
\usepackage{minted}
\hypersetup{
    colorlinks=true,
    linkcolor=blue,
    filecolor=magenta,      
    urlcolor=cyan,
}

\title{Requisitos da Fase 3 do Trabalho Prático}
\author{Matheus Cruz, Martim Kowalczuk Presser}
\date{Turma B}

\begin{document}
\maketitle

\tableofcontents
\newpage


% Início da seção de requisitos

\section{Lista de Requisitos:}

% FASE 1
\subsection{Requisitos Fase 1:}

\subsubsection{Requisitos funcionais:}

\begin{enumerate}
    \item Gerar som a partir da entrada de texto do usuário;
    \begin{enumerate}
        \item[1.1] A entrada de texto deve ser mapeada para uma sequência de notas a serem tocadas e comandos de alteração das mesmas;
        \item[1.2] O som gerado pelo usuário é a sequencia de notas tocada por algum instrumento, em algum volume, oitava e frequência;
        \item[1.3] Os comandos da entrada de texto devem permitir modificar o instrumento (timbre), frequência, volume e oitavas das notas a serem tocadas;
        \item[1.4] O timbre das notas a serem tocadas representam instrumentos diferentes;
        \item[1.5]  A oitava (pitch) das notas é alterado a partir de comandos na entrada;
    \end{enumerate}
    
    \item O software necessita ter um interface gráfica para se comunicar com o usuário;
    \begin{enumerate}
        \item[2.1] Deve conter instruções claras da utilização do software na tela inicial;
        \item[2.2] Deve existir um campo em que o usuário possa digitar sua entrada de texto;
        \item [2.3] Deve existir um botão que o usuário aperte e possa escolher um arquivo de texto para inserir;
        \item[2.4] Deve existir um botão que o usuário aperte e toque o som correspondente ao texto inserido;
        \item[2.5] Se o som estiver tocando e for apertado o botão de tocar, o som é pausado;
        \item[2.6] Se o som estiver pausado, e o botão de tocar for pressionado novamente, o som continua de onde parou;
    \end{enumerate}

    \item O usuário deve ser capaz entender o funcionamento do programa a partir das informações da tela;
\end{enumerate}

\subsubsection{Requisitos Não-Funcionais:}
    \begin{enumerate}
        \item O software deve ser implementado como aplicativo web;
        \begin{enumerate}
            \item[1.1] Necessita de conexão com a internet e um navegador moderno para que o software funcione;
        \end{enumerate}
    \end{enumerate}

% FASE 3
\newpage

\subsection{Requisitos Fase 3:}
\subsubsection{Requisitos Funcionais:}

\begin{enumerate}
    \item Gerar som a partir da entrada de texto e/ou inserção de arquivo feita pelo usuário;
    \begin{enumerate}
        \item[1.1] A entrada de texto e/ou arquivo deve ser mapeada para uma sequência de notas a serem tocadas e comandos de alteração das mesmas;
        \item[1.2] O som gerado pelo usuário é a sequencia de notas tocada por algum instrumento, em algum volume e oitava;
        \item[1.3] Os comandos da entrada de texto e/ou arquivo devem permitir modificar o instrumento (timbre), volume e oitavas das notas a serem tocadas;
        \item[1.4] O timbre das notas a serem tocadas representam instrumentos diferentes;
        \item[1.5] As oitava das notas é alterado a partir de comandos na entrada;
    \end{enumerate}
    
    \item Ter a possibilidade de salvar um arquivo tipo MIDI referente ao som gerado a partir do texto;
    
    \item O software necessita ter um interface gráfica para se comunicar com o usuário;
    \begin{enumerate}
        \item[2.1] Deve conter instruções claras da utilização do software na tela inicial;
        \item[2.2] Deve existir um campo em que o usuário possa digitar sua entrada de texto;
        \item [2.3] Deve existir um botão que o usuário aperte e possa escolher um arquivo de texto para inserir;
        \item[2.4] Deve existir um botão que o usuário aperte e toque o som correspondente ao texto e/ou arquivo inserido;
    \end{enumerate}

    \item O usuario deve ser capaz enteder o funcionamento do programa a partir das informações da tela;
\end{enumerate}
    
\subsubsection{Requisitos Não-Funcionais:}
    \begin{enumerate}
        \item Os usuários devem entender o funcionamento do aplicativo assim que lerem as instruções;
        \item O software deve ser implementado como aplicativo web;
        \begin{enumerate}
            \item[1.1] Necessita de conexão com a internet e um navegador moderno para que o software funcione;
        \end{enumerate}
    \end{enumerate}

% Término da seção de requisitos


\newpage


% Início da seção de código

\section{Definição de Classes:}

O software foi desenvolvido na linguagem Typescript, com o framework React
utilizando Html e Css para construção da interface. Typescript é uma linguagem que é "transpilada" para javascript, e é praticamente uma versão com tipos algébricos e null do javascript. 

O framework React foi escolhido para gerenciar a lógica de desenho e atualização dos componentes (um componente é algo representável como um elemento html) na tela. De maneira simplificada, o framework funciona fornecendo uma uma classe base React.Componente, que é superclasse de todas as classes que são componentes. Os subcomponentes sobrescrevem o método render() de React.Component, e então podem ser utilizados de maneira similar a qualquer outro elemento html, recebendo parâmetros como atributos de elementos html. 

Por exemplo, nosso componente SoundGeneratorButton (definido abaixo) pode ser utilizado dentro de outros elementos html da seguinte maneira:

\begin{minted}{html}
<div>
    <p>Esse é um componente html normal</p>
    <SoundGeneratorButton input="ABCDEFG"></SoundGeneratorButton>
</div>

\end{minted}

\subsection{Classes e Métodos:}
\subsubsection{App.tsx}

Aqui declaramos o componente principal do aplicativo, e no contexto do projeto pode ser considerado fortemente equivalente ao método main() de outras linguagens de programação. 


\begin{minted}{typescript}

class App extends React.Component{
    // A sintaxe de declarão de variaveis em typescript é 'variavel : TIPO'

    // Representa a caixa de entrada de texto do usuário
    private string userTextInput;
    
    // Representa o componente principal que contem a lógica de transformar
    // a entrada em áudio midi
    private SoundGeneratorButton soundGeneratorButton;
    
    // Sobreescrito de React.Componente, define o html emitido pelo
    // componente na tela e desenha o soundGeneratorButton
    public void render()
    
    // Construtor da classe. Só chama o construtor do pai herdado
    public constructor() : App

}

\end{minted}


\subsubsection{SoundGeneratorButton.tsx}
\begin{minted}{typescript}
class SoundGeneratorButton extends React.Component{
  // Representa a entrada do usuário
  private input : string
  
  // Tem a responsabilidade de transformar a entrada do usuário de texto para alguma 
  // representação intermediaria 
  private inputParser: MusicInputParser
  
  // Tem a responsabilidade de transformar a representação intermediaria em um 
  // arquivo em memoria e de serializar a mesma.
  // Não é relacionada com tocar o arquivo Midi
  private midiGenerator: MidiGenerator
  
  // Classe que tem a responsabilidade de tocar um arquivo midi
  // representado como bytes em memoria. Vem da biblioteca MIDIJS
  private const MIDIjs : MidiJs

  // Emite o código html que representa o componente. 
  public void render()
  
  // Os métodos abaixo gerenciam os eventos que
  // são chamados ao pressionar cada um dos botões da interface
  private void onPlaybuttonClick()
  private void onDownloadButtonClick()
  private void onUploadButtonClick()
  
  public constructor() : SoundGeneratorButton


}

\end{minted}

\subsubsection{MidiInstrument.ts}

MidiInstrument é um arquivo que define o tipo de diversos objetos utilizados no programa. A ideia é tentar modelar nosso problema em código utilizando tipos. Não é exatamente uma classe, mas como serve como definição para as classe seguintes, é importante que esteja documentada aqui.

\begin{minted}{typescript}

  // Representa os possíveis instrumentos e seus códigos
  enum MidiInstrument {
    "acoustic-piano" = 0,
    "brtacou-piano" = 1,
    "elecgrand-piano" = 2,
    "honky-tonk-piano" = 3,
    "elec.piano-1" = 4,
    "elec.piano-2" = 5,
    ......
    "telephone" = 124,
    "helicopter" = 125,
    "applause" = 126,
    "gunshot" = 127,
  }

  // O caractere '|' representa um tipo de "Soma"
  // Algo do tipo A | B é ou do tipo A, ou do tipo B.
  type Octave = 1 | 2 | 3 | 4 | 5 | 6 | 7 | 8;
  type Pitch = "A" | "B" | "C" | "D" | "E" | "F" | "G";

  // Representa uma nota a ser tocada
  type MidiNote = {
    type: "NOTE";
    pitch: Pitch;
  };

  // Representa um possível comando que não é uma nota
  type Command =
    | { type: "REPEAT_LAST_OR_SILENCE" }
    | { type: "DOUBLE_VOLUME" }
    | { type: "INCREASE_OCTAVE" }
    | { type: "CHANGE_INSTRUMENT"; value: MidiInstrument }
    | { type: "ADD_TO_INSTRUMENT_NUMBER"; value: number };

  // Representa uma entrada do usuário já parseada
  // representada como uma união de MidiNote e Command
  // Utilizamos esse tipo para representar nossa representação
  // intermediaria de um "Evento" no midi.
  type SoundEvent = MidiNote | Command;

\end{minted}

\subsubsection{MusicInputParser.ts}

\begin{minted}{typescript}
  class MusicInputParser{

  // Transforma a entrada do usuário em uma lista de 
  // soundEvents. Cada caractere da entrada do usuario é mapeado
  // para um soundEvent
  public parseInput(input: string): SoundEvent[]

  // Transforma um caractere para seu mapeamento em Command.
  private parseCommand(char: string): Command

  // Transforma um caractere para seu mapeamento em Nota.
  private parseNote(char: string): Note

  public constructor(): MusicInputParser
}
\end{minted}

\subsubsection{MidiGenerator.ts}
\begin{minted}{typescript}
class MidiGenerator{
  private instrument: MidiInstrument;
  private octave: Octave;
  private lastEvent: SoundEvent | null;
  private volume: number;
  
  // Gera um arquivo Midi pronto para ser serializado 
  // em formato binario. A classe MidiWriter.Track Vem da biblioteca MidiWriterJs
  public MidiWriter.Track generateMidiFromSoundEvents();
  
  public constructor() : MidiGenerator
}
\end{minted}


\subsection{Funcionamento do software}
\begin{enumerate}
\item {Usuário digita o texto no campo de entrada, ou faz upload de algum um arquivo}
\item {Ao clicar em tocar, o botão chama o método onPlaybuttonClick que:}
\subitem {2.1 - Chama o método inputParser.parseInput(input) com a entrada do usuário como parâmetro}
\subitem {2.2 - Utiliza o resultado da chamada anterior como parâmetro do método midiGenerator.generateMidiFromSoundEvents}
\subitem {2.3 - Utiliza o resultado da chamada anterior como parâmetro do método MIDIJS.play}
\subitem {2.4 - MIDI.play toca o arquivo midi para o usuário}
\end{enumerate}


\newpage
\section{Interface Gráfica (GUI):}

\subsection{Justificativa de Layout da Fase 3:}

\paragraph{}O tipo de layout final foi escolhido com o intuito de ter uma interface utilizando componentes padrões para web, partindo da premissa de que os usuários acessam vários sites diferentes e têm contato com interfaces que seguem o mesmo padrão em alguns deles. A ideia é que o usuário consiga entender o funcionamento do website partindo das instruções na tela e de seu entendimento sobre o funcionamento de outros websites que já tenha visitado.

\paragraph{} Para definir o estilo de nossos componentes, Utilizamos a biblioteca de estilos Bootstrap.css em nosso sistema para tentar manter esse padrão. Ela é de fácil entendimento, por ser simples e separar bem os elementos de maneira intuitiva.

\vspace{0.3cm}

\begin{figure}[htp]
    \centering
    \includegraphics{interface-tcp-2020.png}
    \caption{Interface Gráfica do Software da Fase 1}
    \label{fig:interfacef1}
\end{figure}

\newpage

\begin{figure}[htp]
    \centering
    \includegraphics[scale=0.55]{interface-fase-3.png}
    \caption{Interface Gráfica do Software da Fase 3}
    \label{fig:interfacef3}
\end{figure}

\newpage

\section{Descrição dos Procedimentos de Testes:}

\paragraph{}Foram usados testes automatizados em javascript usando o framework Jest, e os arquivos referentes se encontram neste mesmo zip. Os testes são referentes às entradas de texto e as mudanças de eventos descritas no mapeamento. Por exemplo, aumentar a oitava de uma nota quando há o caractere '?' está presente no texto, dobrar o valor do volume de áudio ao acharmos o caractere ' ' no meio do texto etc.

\paragraph{}Testamos separadamente a lógica referente a transformação da entrada para uma lista de SoundEvents (representação intermediaria) e a lógica referente a escrita dos sounds events no arquivo Midi.

\vspace{0.3cm}

\begin{figure}[htp]
    \centering
    \includegraphics[scale=0.7]{testcases.png}
    \caption{Tela de Execução dos Casos de Teste}
    \label{fig:interfacef3}
\end{figure}

\newpage

\bibliography{bib.bib}

\begin{itemize}
\item {midi-writer-js: \url{https://github.com/grimmdude/MidiWriterJS}}
\item {ReactJs: \url{https://pt-br.reactjs.org}}
\item MidiJs \url{https://www.midijs.net/}
\item Jest \url{https://jestjs.io/}
\item Bootstrap \url{https://getbootstrap.com/docs/4.5/getting-started/introduction/}

\end{itemize}

\end{document}
